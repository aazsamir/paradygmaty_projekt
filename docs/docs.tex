\documentclass[12pt,a4paper]{article}

% Pakiety
\usepackage[utf8]{inputenc} 
\usepackage{amsmath, amsfonts, amssymb, polski, indentfirst, graphicx, enumerate}
\usepackage[colorlinks=true, allcolors=blue]{hyperref}
\usepackage{pgfplots}

% Pakiety do formatowania kodu
\usepackage{minted}

% Dane autora
\title{Języki i paradygmaty programowania\\Optymalizacja zapytań za pomocą języka Prolog}
\author{Samir Al-Azazi}
\date{\vspace{-5ex}} % omijamy datę w tytule i usuwamy margines
\newcommand{\studentid}{w66045}

% Strona tytułowa
\begin{document}
\maketitle
\centerline{\studentid}
\begin{figure}[!h]
    \centering
    \includegraphics{./images/wsiz-logo}
\end{figure}
\centerline{\textbf{Wyższa Szkoła Informatyki i Zarządzania}}
\centerline{\textbf{Kolegium Informatyki}}
\vspace*{\fill}
\centerline{\textbf{Rzeszów \today}} % data na dole strony
\thispagestyle{empty} % ukrywa numerację dla strony tytułowej
\clearpage

% Spis treści
\tableofcontents
\pagenumbering{arabic} % przywraca numerację stron, liczy od spisu treści
\clearpage

\section{Wstęp}

Język Prolog\footnote{\href{https://pl.wikipedia.org/wiki/Prolog\_(j\%C4\%99zyk\_programowania)}{https://pl.wikipedia.org/wiki/Prolog\_(j\%C4\%99zyk\_programowania)}} pozwala w prosty sposób rozwiązywać problemy związane z logiką. Pozwala również na wydajne działania na dużych zbiorach danych.\\
W tym projekcie, wykorzystamy język Prolog do optymalizacji zapytań do bazy danych.\\
W tym celu, wykorzystamy SQLite\footnote{\href{https://pl.wikipedia.org/wiki/SQLite}{https://pl.wikipedia.org/wiki/SQLite}} i użyjemy bazy wiedzy Prolog do zapisania danych w postaci cache\footnote{\href{https://pl.wikipedia.org/wiki/Pami\%C4\%99\%C4\%87\_podr\%C4\%99czna}{https://pl.wikipedia.org/wiki/Pami\%C4\%99\%C4\%87\_podr\%C4\%99czna}}. Pozwoli to na uniknięcie wykonywania dodatkowych zapytań do bazy danych, co jest częstym problemem w aplikacjach webowych z mechanizmami mapowania obiektowo-relacyjnego (ORM\footnote{\href{https://pl.wikipedia.org/wiki/Mapowanie\_obiektowo-relacyjne}{https://pl.wikipedia.org/wiki/Mapowanie\_obiektowo-relacyjne}}). 

\section{Implementacja}

\subsection{Aplikacja serwerowa}

Aplikacja serwerowa została napisana w języku Golang\footnote{\href{https://go.dev/}{https://go.dev/}}, w oparciu o narzędzia biblioteki standardowej.\\
Za pomocą interfejsu HTTP, aplikacja wystawia punkty końcowe (ang. endpoints\footnote{\href{https://en.wikipedia.org/wiki/Endpoint\_interface}{https://en.wikipedia.org/wiki/Endpoint\_interface}}) do komunikacji REST\footnote{\href{https://pl.wikipedia.org/wiki/Representational\_state\_transfer}{https://pl.wikipedia.org/wiki/Representational\_state\_transfer}}.\\:

\begin{itemize}
    \item \textbf{GET /api/users}
    \item \textbf{POST /api/users?name=\{name\}}
    \item \textbf{POST /tags?user\_id=\{user\_id\}\&name=\{name\}}
    \item \textbf{DELETE /tags?user\_id=\{user\_id\}}
\end{itemize}

Za ich pomocą użytkownik może pobrać użytkowników, wraz z przypisanymi do niego tagami, dodać nowego użytkownika, dodać tag do użytkownika oraz usunąć wszystkie tagi użytkownika.\\

Zapytania do bazy danych zostały sztucznie opóźnione, by zasymulować opóźnienia w komunikacji sieciowej, używając wzorca projektowego \textbf{Decorator}\footnote{\href{https://refactoring.guru/design-patterns/decorator}{https://refactoring.guru/design-patterns/decorator}}.\\

\subsection{Implementacja Prolog}

Został przygotowany program w języku Prolog, w pliku \textbf{curator.pl}.
\begin{minted}[linenos]{prolog}
:- set_prolog_flag(verbose, silent).
:- initialization main.
head([H|_], H).
main :-
    current_prolog_flag(argv, Argv),
    head(Argv, Who),
    atom_number(Who, WhoNum),
    related(WhoNum, X),
    format('~w', X),
    halt.
main :-
    format('0'),
    halt.
\end{minted}

\clearpage
Wyłączamy wypisywanie dodatkowych informacji po interpretacji programu.
\begin{minted}[linenos, firstnumber=1]{prolog}
    :- set_prolog_flag(verbose, silent).
\end{minted}

Oznaczamy predykat \textbf{main} jako punkt wejścia programu.
\begin{minted}[linenos, firstnumber=2]{prolog}
    :- initialization main.
\end{minted}

Definiujemy predykat \textbf{head}, który zwraca pierwszy element listy.
\begin{minted}[linenos, firstnumber=3]{prolog}
    head([H|_], H).
\end{minted}

Definiujemy predykat \textbf{main}, który jest wywoływany przy uruchomieniu programu.
\begin{minted}[linenos, firstnumber=4]{prolog}
    main :-
\end{minted}

Pobieramy listę argumentów wywołania programu.
\begin{minted}[linenos, firstnumber=4]{prolog}
    current_prolog_flag(argv, Argv),
\end{minted}

Pobieramy pierwszy argument, używając utworzonego wcześniej predykatu \textbf{head}.
\begin{minted}[linenos, firstnumber=4]{prolog}
    head(Argv, Who),
\end{minted}

Konwertujemy argument na atom liczbowy.
\begin{minted}[linenos, firstnumber=4]{prolog}
    atom_number(Who, WhoNum),
\end{minted}

Zapisujemy wynik zapytania \textbf{related} do zmiennej \textbf{X}.
\begin{minted}[linenos, firstnumber=4]{prolog}
    related(WhoNum, X),
\end{minted}

Wypisujemy wynik na wyjście standardowe.
\begin{minted}[linenos, firstnumber=4]{prolog}
    format('~w', X),
\end{minted}

I kończymy działanie programu.
\begin{minted}[linenos, firstnumber=4]{prolog}
    halt.
\end{minted}

Jeśli którykolwiek z powyższych predykatów nie został spełniony, program przechodzi do tego etapu.
\begin{minted}[linenos, firstnumber=4]{prolog}
    main :-
\end{minted}

Wypisujemy 0 na wyjście standardowe.
\begin{minted}[linenos, firstnumber=4]{prolog}
    format('0'),
\end{minted}

I kończymy działanie programu.
\begin{minted}[linenos, firstnumber=4]{prolog}
    halt.
\end{minted}

W momencie gdy dodamy nowy TAG do użytkownika, do bazy danych zostanie dodana nowa informacja.

\begin{minted}[]{prolog}
    related(1,1)
\end{minted}

Informuje o tym, że użytkownik o ID 1 jest w relacji do tagów i aplikacja powinna odpytać o nie bazę danych.\\

Możemy sprawdzić działanie za pomocą terminala, wykonując polecenie
\begin{minted}{bash}
    swipl curator.pl 1
\end{minted}

W ten sam sposób, program jest wywołany z aplikacji serwerowej, gdzie 1 to ID użytkownika.


\clearpage
\section{Pomiary}

\subsection{Natywne zapytanie SQL}

W bazie danych znajduje się 4 użytkowników.

\begin{minted}{json}
    [
        {
            "ID": 1,
            "Name": "John",
            "Tags": null
        },
        {
            "ID": 2,
            "Name": "John",
            "Tags": null
        },
        {
            "ID": 3,
            "Name": "John",
            "Tags": null
        },
        {
            "ID": 4,
            "Name": "John",
            "Tags": null
        }
    ]
\end{minted}

Czas pobrania ich za pomocą zapytania \textbf{GET /api/users} wynosi 6 sekund.\\

\subsection{Zapytania wspomagane przez Prolog}

Po dodaniu dekoracji Prolog do zapytania, czas pobrania 4 użytkowników wynosi 3 sekundy.

\clearpage
\section{Wnioski}

Zastosowanie języka Prolog do optymalizacji zapytań do bazy danych pozwala na znaczne przyspieszenie działania aplikacji przy dużych nakładach czasowych na mapowanie obiektowo-relacyjne.\\

Wadą takiego rozwiązania jest konieczność utrzymywania dodatkowej bazy wiedzy, która musi być aktualizowana w momencie zmian w bazie danych i może powodować problemy z synchronizacją z powodu dwóch źródeł prawdy.\\

Można minimalizować te problemy poprzez zastosowanie mechanizmów automatycznego czyszczenia bazy wiedzy co dany interwał czasowy, lub w momencie wykrycia zmian w bazie danych.\\

Zbudowanie generalnego mechanizmu, który można zintegrować z różnymi narzędziami może przynieść znaczne korzyści, gdzie występuje problem N+1 \footnote{\href{https://stackoverflow.com/questions/97197/what-is-the-n1-selects-problem-in-orm-object-relational-mapping}{https://stackoverflow.com/questions/97197/what-is-the-n1-selects-problem-in-orm-object-relational-mapping}}.

\section{Linki}
\href{https://github.com/aazsamir/paradygmaty_projekt}{GitHub}

\end{document}